\documentclass[a4paper,12pt]{article}

\usepackage[a4paper, total={6in, 9.5in}]{geometry}
\usepackage[utf8]{inputenc}
\usepackage[T1]{fontenc}
\usepackage[british]{babel}

\usepackage{graphicx}
\graphicspath{{images/}}

\usepackage{booktabs}
\usepackage{longtable}

\usepackage{xcolor}
\usepackage{listings}
\lstset{basicstyle=\ttfamily,
  showstringspaces=false,
  commentstyle=\color{red},
  keywordstyle=\color{blue}
}

\usepackage{mathtools}
\usepackage{amssymb}
\usepackage{enumitem}
\usepackage{lastpage}

\usepackage{fancyhdr}
\pagestyle{fancy}
\lhead{Sistemas Basados en el conocimiento}
\rhead{Page \thepage\ of \pageref{LastPage}}
\cfoot{\scriptsize{\today{}}}

\begin{document}

% First page %%%%%%%%%%%%%%%%%%%%%%%%%%%%%%%%%%%%%%%%%%%%%%%%%%%%%%%%%%%%%%
\begin{titlepage}
\begin{center}

\includegraphics[width=0.6\textwidth]{logoesi}\\[5cm]

% Title
\rule{\linewidth}{0.5mm} \\[0.4cm]
{ \huge \bfseries Sistemas Basados en el Conocimiento\\[0.4cm] }
\rule{\linewidth}{0.5mm} \\[1.5cm]
{\huge Diagnóstico de problemas mecánicos (2017/2018)}\\[0.5cm]

% Author
\large{Juan Jos\'e Corroto Mart\'in}

\end{center}
\end{titlepage}

\tableofcontents
\newpage

\section{Descripción y alcance}
\subsection{Descripción del proyecto}
Se desarrollará un prototipo de sistema capaz de diagnosticar el problema que tiene nuestro coche, en base a una serie de "síntomas" que podemos percibir, como por ejemplo que salga humo blanco por el tubo de escape. Este sistema estaría principalmente enfocado al aprendizaje para futuros mecánicos, y para la ayuda a la decisión en caso de que el mecánico sea poco experimentado.

\subsection{Alcance y límites del proyecto}
El sistema tratará de diagnosticar problemas con automóviles convencionales con un motor de cuatro tiempos. Sólo se tendrán en cuenta problemas relacionados con fallos en el turbocompresor del coche y válvula E.G.R. No se tendrán en cuenta problemas con las ruedas, chapa y pintura, cristales o sistema eléctrico.

\section{Estudio de viabilidad}
Para poder justificar el la realización de este sistema, se va a realizar un estudio de viabilidad basado en el \texttt{test de Slagel}.
Éste test consiste en la evaluación del sistema en torno a \textbf{cuatro dimensiones}:
\begin{itemize}
\item Plausibilidad.
\item Justificación.
\item Adecuación.
\item Éxito.
\end{itemize}

De cada dimensión se analizarán una serie de características y a cada una se le asignará un valor. Cada una de estas características tiene un peso definido por el propio test. Una vez analizadas todas las características se realizará una pseudo-media geométrica para obtener el valor asociado a dicha dimensión.

La leyenda de las abreviaturas que se van a usar es:
\begin{itemize}
\item CAT: categoría. Puede ser EX (Expertos), TA (Tarea), DU (Directivos y/o usuarios).
\item IDEN: Identificador de la característica.
\item Tipo: E (Esencial), D (Deseable). Si una característica esencial no alcanza de valor un 7, el sistema será rechazado.
\end{itemize}

\newpage

\begin{table}[]
\centering
\caption{Plausibilidad}
\label{Plausibilidad}
\resizebox{\textwidth}{!}{%
\begin{tabular}{@{}llllll@{}}
\toprule
CAT & IDEN & Peso & Valor & Denominación de la Característica                                                                                                                                                                                                                                                  & TIPO \\ \midrule
EX  & P1   & 10   & 10    & Existen expertos.                                                                                                                                                                                                                                                                  & E    \\ \midrule
EX  & P2   & 10   & 9     & \begin{tabular}[c]{@{}l@{}}El experto es genuino. Sí, aunque no sea un mecánico \\ reconozido a nivel mundial, el experto disponible sí \\ que es reconocido a nivel local.\end{tabular}                                                                                           & E    \\ \midrule
EX  & P3   & 8    & 9     & \begin{tabular}[c]{@{}l@{}}El experto es cooperativo. Sí, el único problema \\ que podría haber es que es un mecánico muy \\ solicitado, y para hablar con él hay\\ que avisarle con anterioridad.\end{tabular}                                                                    & D    \\ \midrule
EX  & P4   & 7    & 10    & \begin{tabular}[c]{@{}l@{}}El experto es capaz de articular sus métodos, \\ pero no categoriza.\end{tabular}                                                                                                                                                                       & D    \\ \midrule
TA  & P5   & 10   & 8     & \begin{tabular}[c]{@{}l@{}}Existen suficientes casos de prueba: normales, \\ típicos. ejemplares, correosos, etc.\end{tabular}                                                                                                                                                     & E    \\ \midrule
TA  & P6   & 10   & 10    & La tarea está bien estructurada y se entiende.                                                                                                                                                                                                                                     & D    \\ \midrule
TA  & P7   & 10   & 7     & \begin{tabular}[c]{@{}l@{}}Sólo requiere habilidad cognoscitiva. Sí puesto que \\ el sistema solo se encarga de dar el diagnóstico, \\ no de resolver el problema. Puede ser que para \\ conseguir ciertos síntomas deseados sean \\ necesarias ciertas herramientas.\end{tabular} & D    \\ \midrule
TA  & P8   & 9    & 10    & \begin{tabular}[c]{@{}l@{}}No se precisan resultados óptimos, sino sólo \\ satisfactorios, sin comprender el proyecto.\end{tabular}                                                                                                                                                & D    \\ \midrule
TA  & P9   & 9    & 10    & La tarea no requiere sentido común.                                                                                                                                                                                                                                                & D    \\ \midrule
DU  & P10  & 7    & 10    & \begin{tabular}[c]{@{}l@{}}Los directivos están verdaderamente comprometidos \\ con el proyecto.\end{tabular}                                                                                                                                                                      & D    \\ \bottomrule
\end{tabular}%
}
\end{table}

Aplicando la fórmula de pseudo media geométrica adaptada a la dimensión de Plausibilidad:

$$VCP = \prod_{i=1,2,5} (V_{pi}//V_{ui}) [(\prod_{i=1}^{10} P_pi*V_pi)]^{1/10}$$

Obtenemos como resultado aproximado 82.

\newpage

\begin{table}[]
\centering
\caption{Justificación}
\label{Justificación}
\resizebox{\textwidth}{!}{%
\begin{tabular}{@{}llllll@{}}
\toprule
CAT & IDEN & Peso & Valor & Denominación de la Característica                                                                                                                                                                                                          & TIPO \\ \midrule
EX  & J1   & 10   & 9     & \begin{tabular}[c]{@{}l@{}}El experto NO está disponible. Puesto que \\ el sistema está diseñado para el aprendizaje,\\ se considera que no hay experto disponible\end{tabular}                                                            & E    \\ \midrule
EX  & J2   & 10   & 9     & \begin{tabular}[c]{@{}l@{}}Hay escasez de experiencia humana. De nuevo,\\ se puede enconstrar culquier mecánico, pero \\ como la aplicación está orientada al aprendizaje,\\ se considera que no se acude a ningún experto.\end{tabular}   & D    \\ \midrule
TA  & J3   & 8    & 10    & \begin{tabular}[c]{@{}l@{}}Existe necesidad de experiencia simultánea en\\ muchos lugares. El experto puede querer un ayudante.\\ Con este sistema no tendría necesidad de contratar un \\ ayudante con su misma experiencia.\end{tabular} & D    \\ \midrule
TA  & J4   & 10   & 8     & \begin{tabular}[c]{@{}l@{}}Necesidad de experiencia en entornos hostiles penosos \\ y/o poco gratificantes. La mecánica puede ser un ambiente\\ poco gratificante.\end{tabular}                                                            & E    \\ \midrule
TA  & J5   & 8    & 7     & No existen soluciones alternativas admisibles                                                                                                                                                                                              & E    \\ \midrule
DU  & J6   & 7    & 8     & Se espera una alta tasa de recuperación de la inversión.                                                                                                                                                                                   & D    \\ \midrule
DU  & J7   & 8    & 9     & Resuelve una tarea útil y necesaria.                                                                                                                                                                                                       & E    \\ \bottomrule
\end{tabular}%
}
\end{table}

Aplicando la fórmula de pseudo media geométrica adaptada a la dimensión de Justificación:

$$VCJ = \prod_{i=1,4,5,7} (V_{ji}//V_{ui}) [(\prod_{i=1}^{7} P_ji*V_ji)]^{1/7}$$

Obtenemos como resultado aproximado 72.

\newpage

\begin{table}[]
\centering
\caption{Adecuación}
\label{Adecuación}
\resizebox{\textwidth}{!}{%
\begin{tabular}{@{}llllll@{}}
\toprule
CAT & IDEN & Peso & Valor & Denominación de la Característica                                                                                                                                               & TIPO \\ \midrule
EX  & A1   & 5    & 9     & La experiencia del experto está poco organizada.                                                                                                                                & D    \\ \midrule
TA  & A2   & 6    & 9     & Tiene valor prácitco.                                                                                                                                                           & D    \\ \midrule
TA  & A3   & 7    & 10    & Es una tarea más práctica que estratégica.                                                                                                                                      & D    \\ \midrule
TA  & A4   & 7    & 9     & \begin{tabular}[c]{@{}l@{}}La tarea da soluciones que sirvan a necesidades\\ a largo plazo.\end{tabular}                                                                        & E    \\ \midrule
TA  & A5   & 5    & 8     & \begin{tabular}[c]{@{}l@{}}La tarea no es demasiado fácil, pero es de\\ conocimiento intensivo tanto propio del dominio,\\ como de manipulación de la información.\end{tabular} & D    \\ \midrule
TA  & A6   & 6    & 10    & \begin{tabular}[c]{@{}l@{}}Es de tamaño manejable, y/o es posible un enfoque\\ gradual y/o, una descomposición en subtareas \\ independientes.\end{tabular}                     & D    \\ \midrule
EX  & A7   & 7    & 10    & \begin{tabular}[c]{@{}l@{}}La transferencia de experiencia entre humanos es \\ factible(experto a aprendiz).\end{tabular}                                                       & E    \\ \midrule
TA  & A8   & 6    & 4     & \begin{tabular}[c]{@{}l@{}}Estaba identificada como un problema en el área\\ y los efectos de la introducción de un SE pueden\\ planificarse.\end{tabular}                      & D    \\ \midrule
TA  & A9   & 9    & 10    & No requiere respuestas en tiempo real "inmediato".                                                                                                                              & E    \\ \midrule
TA  & A10  & 9    & 10    & La tarea no requiere investigación básica.                                                                                                                                      & E    \\ \midrule
TA  & A11  & 5    & 10    & \begin{tabular}[c]{@{}l@{}}El experto usa básicamente razonamiento simbólico\\ que implica factores subjetivos.\end{tabular}                                                    & D    \\ \midrule
TA  & A12  & 5    & 10    & Es esencialmente de tipo heurístico.                                                                                                                                            & D    \\ \bottomrule
\end{tabular}%
}
\end{table}

Aplicando la fórmula de pseudo media geométrica adaptada a la dimensión de Adecuación:

$$VCA = \prod_{i=4,7,9,10} (V_{ai}//V_{ui}) [(\prod_{i=1}^{12} P_ai*V_ai)]^{1/12}$$

Obtenemos como resultado aproximado 55.

\newpage

\begin{longtable}[c]{@{}llllll@{}}
\caption{Éxito}
\label{Éxito}\\
\toprule
CAT & IDEN & Peso & Valor & Denominación de la Característica                                                                                                                                                                                                                                                                                                                     & TIPO \\* \midrule
\endfirsthead
%
\endhead
%
EX  & E1   & 8    & 9     & \begin{tabular}[c]{@{}l@{}}No se sienten amenazados por el proyecto\\  son capaces de sentirso intelectualmente\\  unidos al proyecto.\end{tabular}                                                                                                                                                                                                   & D    \\* \midrule
EX  & E2   & 6    & 10    & \begin{tabular}[c]{@{}l@{}}Tienen un brillante historial en la \\ realización de esta tarea.\end{tabular}                                                                                                                                                                                                                                             & D    \\* \midrule
EX  & E3   & 5    & 10    & \begin{tabular}[c]{@{}l@{}}Hay acuerdos en lo que constituye una\\ buena solución a la tarea.\end{tabular}                                                                                                                                                                                                                                            & D    \\* \midrule
EX  & E4   & 5    & 10    & \begin{tabular}[c]{@{}l@{}}La única justificación para dar un paso\\ en la solución es la calidad de la solución\\ final.\end{tabular}                                                                                                                                                                                                                & D    \\* \midrule
EX  & E5   & 6    & 10    & \begin{tabular}[c]{@{}l@{}}No hay un plazo de funalización estricto,\\ ni ningún otro proyecto depende de esta\\ tarea. El experto no necesita que este\\ sistema esté finalizado en ningún momento,\\ ni le influye en sus tareas de diario.\end{tabular}                                                                                            & D    \\* \midrule
TA  & E6   & 7    & 10    & No está influenciada por vaivenes políticos.                                                                                                                                                                                                                                                                                                          & E    \\* \midrule
TA  & E7   & 8    & 4     & \begin{tabular}[c]{@{}l@{}}Existen ya SS.EE. que resuelvan esa o\\ parecidas tareas. No tengo constancia de \\ que existan SS.EE. en esta tarea a día de\\ hoy.\end{tabular}                                                                                                                                                                          & D    \\* \midrule
TA  & E8   & 8    & 8     & \begin{tabular}[c]{@{}l@{}}Hay cambios mínimos en los procedimientos\\ habituales. Aunque se inventen nuevas técnicas\\ para la recolección de "síntomas" en los talleres,\\ el procedimiento para el diagnóstico será el mismo,\\ puesto que los motores contemplados siempre son \\ similares.\end{tabular}                                         & D    \\* \midrule
TA  & E9   & 5    & 9     & Las soluciones son explicables o interactivas.                                                                                                                                                                                                                                                                                                        & D    \\* \midrule
TA  & E10  & 7    & 8     & \begin{tabular}[c]{@{}l@{}}La tarea es de I+D de carácter práctico, pero\\ no ambas cosas simlutáneamente.\end{tabular}                                                                                                                                                                                                                               & E    \\* \midrule
DU  & E11  & 6    & 9     & \begin{tabular}[c]{@{}l@{}}Están mentalizados y tienen expectativas \\ realistas tanto en el alcance como en las\\ limitaciones.\end{tabular}                                                                                                                                                                                                         & D    \\* \midrule
DU  & E12  & 7    & 10    & No rechazan de plano esta tecnología.                                                                                                                                                                                                                                                                                                                 & E    \\* \midrule
DU  & E13  & 6    & 9     & \begin{tabular}[c]{@{}l@{}}El sistema interactúa inteligente y \\ amistosamente con el usuario.\end{tabular}                                                                                                                                                                                                                                          & D    \\* \midrule
DU  & E14  & 9    & 10    & \begin{tabular}[c]{@{}l@{}}El sistema es capaz de explicar al usuario\\ su razonamiento.\end{tabular}                                                                                                                                                                                                                                                 & D    \\* \midrule
DU  & E15  & 8    & 6     & \begin{tabular}[c]{@{}l@{}}La inserción del sistema se efectúa sin traumas,\\ es decir, apenas se interfiere en la rutina cotidiana\\ de la empresa. El sistema puede ser algo intrusivo\\ en el caso de uso de un mecánico novato, puesto \\ que normalmente el ordenador donde está el\\ sistema puede estar en una habitación aparte.\end{tabular} & D    \\* \midrule
DU  & E16  & 6    & 10    & \begin{tabular}[c]{@{}l@{}}Están comprometidos durante toda la duración\\ del proyecto, incluso después de su implantación.\end{tabular}                                                                                                                                                                                                              & D    \\* \midrule
DU  & E17  & 8    & 8     & Se efectúa una adecuada transferencia tecnológica.                                                                                                                                                                                                                                                                                                    & E    \\* \bottomrule
\end{longtable}

Aplicando la fórmula de pseudo media geométrica adaptada a la dimensión de Éxito:

$$VCE = \prod_{i=6,10,12,17} (V_{ei}//V_{ui}) [(\prod_{i=1}^{17} P_ei*V_ei)]^{1/17}$$

Obtenemos como resultado aproximado 57.


Una vez hemos estudiado las cuatro dimensiones, se procede a hacer una media aritmética para conseguir el valor de viabilidad que nos aporta el \texttt{test de Slagel}.

$$VC = \sum_{i=1}^4 VC_i / 4$$

Por tanto:

$$VC = \frac{82 + 72 + 55 + 57}{4} = 66.5$$

Para conseguir un valor porcentual, recalculamos las cuatro dimensiones puntuando con el valor máximo a todas las características. Haciendo esto obtenemos el valor: 
$$VC = \frac{89 + 86 + 62 + 67}{4} = 76$$

Una vez tenemos el valor máximo en un \texttt{test de Slagel}, procedemos a calcular el porcentaje de nuestra viabilidad:

$$VC = \frac{66.5}{76} * 100 = 87.5$$

Por tanto, nuestro sistema tiene un $87,5$\% de viablidad según el \texttt{test de Slagel}.


\end{document}

%%% Local Variables:
%%%   coding: utf-8
%%%   flyspell-mode: t
%%%   ispell-local-dictionary: "british"
%%% End:
